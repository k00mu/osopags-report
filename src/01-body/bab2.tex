%-----------------------------------------------------------------------------%
\chapter{\babDua}
\label{bab:2}
%-----------------------------------------------------------------------------%
Penelitian Terkait

Bagian ini membahas beberapa penelitian terkait yang menjadi landasan dan referensi dalam pengembangan proyek akhir ini. Dengan mengkaji penelitian-penelitian sebelumnya, kita dapat memahami perkembangan terkini dalam bidang yang sama dan mengidentifikasi celah yang perlu diisi oleh penelitian ini. Penelitian terkait memberikan konteks yang lebih luas dan membantu mengarahkan penelitian agar lebih relevan dan bermanfaat.

Pengembangan Perangkat Lunak untuk Membantu Evaluasi User Experience dalam Game dengan Metode Heatmaps

Penelitian ini berfokus pada pengembangan perangkat lunak yang bertujuan untuk membantu analisis pengalaman pengguna (user experience) dalam game melalui metode heatmaps. Heatmaps merupakan representasi grafis dari data di mana nilai-nilai individu dalam sebuah matriks direpresentasikan sebagai warna. Dalam konteks penelitian ini, heatmaps digunakan untuk melacak perilaku pengguna selama berinteraksi dengan game, seperti pergerakan cursor, posisi klik, dan pandangan mata.

Metode yang digunakan dalam penelitian ini melibatkan beberapa langkah. Pertama, dilakukan studi literatur untuk memahami teknologi dan metode yang relevan. Selanjutnya, dilakukan perancangan sistem yang mencakup pembuatan antarmuka pengguna, integrasi dengan aplikasi pelacakan mata (GazePointer), serta implementasi fitur perekaman layar dan visualisasi heatmaps.

Pengujian sistem dilakukan dengan menggunakan game “Aquaculture Land” sebagai studi kasus. Data yang dihasilkan mencakup heatmaps yang menunjukkan area dominan yang dilihat dan diklik oleh pengguna. Hasil analisis data ini diharapkan dapat membantu pengembang game dalam meningkatkan kualitas user experience secara keseluruhan.

Penelitian ini menunjukkan bahwa perangkat lunak yang dikembangkan dapat membantu pengembang dalam mengevaluasi user experience secara efektif. Program ini mudah digunakan dalam proses pengambilan dan visualisasi data, serta dapat memberikan wawasan yang berharga untuk meningkatkan desain antarmuka game [6].

Layanan Analytics Terbuka Sebagai Solusi Percepatan Pertumbuhan Game Indie Lokal Untuk Prediksi Scaling dan Pemerataan Pasar Potensial

Penelitian ini bertujuan untuk mengembangkan layanan analytics terbuka yang dirancang untuk mempercepat pertumbuhan game indie lokal dengan memprediksi scaling dan pemerataan pasar potensial. Di Indonesia, meskipun banyak studio game indie yang bermunculan, mereka sering kali mengabaikan pentingnya layanan analytics dalam game mereka untuk memprediksi pasar potensial. Hal ini disebabkan kurangnya layanan analytics yang terbuka dalam komunitas developer indie, di mana mayoritas layanan hanya menyediakan informasi kepada satu akun developer.

Metode yang digunakan dalam penelitian ini mencakup beberapa langkah. Pertama, dilakukan studi literatur untuk memahami teknologi dan metode yang relevan. Selanjutnya, dilakukan perancangan sistem yang mencakup pembuatan REST API, dokumentasi API, serta halaman penyajian data berupa statistik deskriptif dengan informasi Game Playtime Session, D/MAU (Daily/Monthly Active User), dan Device Retention.

Pengujian sistem dilakukan dengan menggunakan game “Ninja Adventure” dari studio indie Vizard Tales sebagai studi kasus. Data yang dihasilkan mencakup informasi mengenai aktivitas pemain yang direkam melalui package dan plugin yang diimplementasikan dalam game engine Unity3D. Hasil analisis data ini diharapkan dapat membantu developer game dalam meningkatkan kualitas dan efektivitas game mereka dengan memanfaatkan informasi yang dibagikan secara terbuka dalam komunitas.

Penelitian ini menunjukkan bahwa sistem analytics yang dikembangkan dapat membantu developer game indie dalam menganalisis aktivitas pemain secara efektif dan berbagi informasi untuk prediksi scaling serta pemerataan pasar potensial. Sistem ini mudah diimplementasikan dan memberikan wawasan yang berharga untuk pengembangan game lebih lanjut [2].

Data Cracker: Developing a Visual Game Analytic Tool for Analyzing Online Gameplay

Penelitian ini membahas pengembangan alat analisis game visual yang dirancang untuk memantau perilaku pemain selama memainkan sesi game online. Alat yang dinamakan Data Cracker ini dikembangkan untuk memonitor gameplay di “Dead Space 2,” sebuah game dari franchise Dead Space. Tujuan utama dari alat ini adalah meningkatkan literasi data tim game dengan cara melibatkan seluruh anggota tim dalam analisis game.

Data Cracker dibangun dengan pendekatan yang berfokus pada visualisasi data melalui berbagai prototipe awal dan branding alat untuk tim Dead Space 2. Alat ini memungkinkan tim untuk memantau berbagai metrik pemain, seperti waktu mulai dan selesai permainan, posisi pemain, jenis senjata yang digunakan, dan lain-lain. Data tersebut dikumpulkan melalui sistem telemetri yang terintegrasi dalam kode game, mengirimkan data ke server untuk kemudian dianalisis dan divisualisasikan pada antarmuka web klien.

Penelitian ini menunjukkan bahwa alat analisis game visual seperti Data Cracker dapat meningkatkan proses desain game dengan memberikan wawasan yang lebih dalam tentang perilaku pemain. Alat ini tidak hanya membantu dalam pengembangan game tetapi juga dalam memelihara dan meningkatkan kualitas game setelah dirilis karena terus memantau dan menganalisis data pemain [8].

Produk Terkait

Bagian ini membahas produk-produk terkait yang memiliki fungsi atau fitur serupa dengan proyek akhir yang sedang dikembangkan. Tujuan dari peninjauan ini adalah untuk memahami kekuatan dan kelemahan produk-produk yang ada, serta untuk mendapatkan inspirasi dan wawasan yang dapat diterapkan dalam pengembangan proyek akhir.

AccelByte

AccelByte adalah platform layanan backend untuk game yang menyediakan berbagai fitur termasuk analytics, manajemen akun, dan penyimpanan data [9]. Komponen analytics yang ditawarkan, AccelByte Metrics, memungkinkan pengembang game untuk mengumpulkan, menyimpan, memproses, dan mengekspor data performa game. Beberapa metrik penting yang disediakan meliputi Daily Active Users (DAU), Monthly Active Users (MAU), retensi pemain, sesi permainan, dan monetisasi.

IronSource

IronSource adalah platform monetisasi dan distribusi aplikasi yang menyediakan solusi komprehensif untuk mengoptimalkan pendapatan dan meningkatkan pertumbuhan pengguna [10]. IronSource Analytics menawarkan pelacakan dan pengukuran kampanye iklan, analytics performa aplikasi, dan pencegahan kecurangan. Platform ini memungkinkan pengembang untuk melacak metrik penting seperti DAU, MAU, sesi permainan, dan retensi pemain. IronSource juga menyediakan alat A/B testing untuk membantu pengembang mengoptimalkan pengalaman pengguna dan meningkatkan pendapatan.

Adjust

Adjust adalah platform analytics yang fokus pada tracking dan pengukuran ads campaign aplikasi mobile [11]. Adjust Analytics menyediakan layanan untuk install/uninstall tracking, application performance analytics, dan fraud prevention. Adjust juga menawarkan fitur custom event analytics dan D/MAU.

Komparasi Produk

Komparasi ini bertujuan untuk melihat bagaimana masing-masing produk terkait beroperasi dalam hal fitur yang ditawarkan, biaya, dan model layanan.

Tabel 2.1. Tabel Komparasi Produk Terkait

Dengan membandingkan fitur dan kelebihan masing-masing produk, proyek akhir ini bertujuan untuk mengintegrasikan keunggulan dari ketiga platform ini, namun dengan menyediakan semua fitur utama secara gratis, sehingga dapat diakses oleh pengembang game dan publik tanpa beban biaya tambahan.

Teori Penunjang

Bagian ini membahas teori-teori yang mendasari penelitian dan pengembangan proyek akhir ini. Teori-teori ini memberikan landasan ilmiah dan metodologis yang kuat, serta membantu dalam memahami konteks dan penerapan berbagai konsep yang digunakan dalam proyek ini.

Game Analytics

Game analytics adalah proses mengidentifikasi dan menyampaikan pola-pola penting yang dapat digunakan sebagai dasar untuk membuat keputusan strategis yang lebih baik dalam pengembangan dan manajemen permainan [12]. Tujuan dari game analytics adalah untuk memecahkan masalah, membuat prediksi dalam bisnis game, membantu pengambilan keputusan, mempromosikan tindakan optimasi, dan meningkatkan kinerja bisnis permainan.

Gambar 2.1. Pecahan Game Analytics

Seperti pada gambar di atas, Game Analytics terbagi menjadi dua jenis, yaitu Game Value Chain dan Orthogonal to Value Chain.

Game Value Chain

Game Player Analytics

Melibatkan analisis terhadap pemain game, termasuk segmentasi pemain, perilaku pemain, gameplay, antarmuka, sistem, proses, dan performa.

Game Development Analytics

Fokus pada analisis selama pengembangan game, termasuk sistem yang digunakan, proses pengembangan, dan performa game selama tahap pengembangan.

Game Publishing Analytics

Menganalisis aspek penerbitan game, termasuk akuisisi pemain, retensi pemain, dan pendapatan yang dihasilkan dari game.

Distribution Channel Analytics

Menganalisis saluran distribusi game untuk memahami bagaimana game didistribusikan dan diterima oleh pasar.

Orthogonal to Value Chain

Game Prediction

Meliputi prediksi seperti churn prediction (prediksi pemain yang akan berhenti bermain) dan revenue prediction (prediksi pendapatan).

Data Visualization

Fokus pada visualisasi data untuk memudahkan pemahaman dan interpretasi data analytics yang telah dikumpulkan.

Fitur Analytics yang akan dikerjakan

A/B Testing

A/B testing memungkinkan pengembang membuat keputusan berdasarkan data untuk mengoptimalkan aplikasi mereka. Proses ini melibatkan pembandingan dua varian (A dan B) dengan pengguna yang ditugaskan secara acak untuk menghindari bias. Tujuan eksperimen harus ditentukan dengan jelas untuk mengukur KPI yang relevan, seperti retensi pemain dan tingkat konversi. Analisis hasil A/B testing membantu memahami dampak dari perubahan yang diuji terhadap metrik kinerja utama dan meningkatkan pengalaman pengguna secara keseluruhan [13].

Daily/Monthly Active User Tracking

Daily Active User (DAU) dan Monthly Active User (MAU) adalah metrik penting dalam game analytics yang digunakan untuk mengukur jumlah pengguna unik yang berinteraksi dengan game dalam periode harian dan bulanan. Metrik ini memberikan wawasan tentang keterlibatan pemain dan popularitas game, yang sangat penting untuk strategi pengembangan dan pemasaran game [8].

Custom Event Tracking

Custom event tracking adalah teknik dalam game analytics yang memungkinkan pengembang untuk melacak interaksi spesifik pengguna dengan berbagai elemen dalam game yang tidak tercakup oleh metrik standar. Dengan menggunakan custom event tracking, pengembang dapat mengumpulkan data yang lebih terperinci tentang perilaku pemain dan mengidentifikasi area untuk perbaikan dan optimasi [14].

User Acquisition Tracking

User Acquisition (UA) tracking adalah proses mengukur efektivitas kampanye pemasaran dengan melacak asal pengguna baru yang berinteraksi dengan game. Metrik ini sangat penting untuk memahami dari mana pemain berasal, platform mana yang paling efektif, dan strategi pemasaran apa yang memberikan hasil terbaik [15].

User Playtime Session Tracking

User Playtime Session Tracking adalah proses pengumpulan dan analisis data terkait durasi dan frekuensi sesi bermain pengguna dalam sebuah game. Metrik ini sangat penting untuk memahami bagaimana pemain berinteraksi dengan game dan untuk mengidentifikasi pola bermain yang dapat digunakan untuk meningkatkan desain game dan pengalaman pengguna [15].

Client-Server Architecture

Client-Server Architecture adalah sebuah model arsitektur yang memisahkan tugas antara penyedia layanan (server) dan pemohon layanan (client) [16]. Server bertanggung jawab untuk menyimpan, memproses, dan mengelola data atau layanan, sementara klien mengirimkan permintaan ke server dan menampilkan hasil kepada pengguna. Arsitektur ini memungkinkan distribusi beban kerja yang lebih efisien, pemeliharaan yang lebih mudah, dan skalabilitas yang lebih baik dalam sistem jaringan modern.

Open Source Software

Open Source Software didefenisikan dari model lisensi yang memberikan pengguna kebebasan untuk menjalankan program untuk tujuan apapun, mempelajari dan memodifikasi source code, serta mendistribusikan kembali versi asli maupun yang telah dimodifikasi dari perangkat lunak tersebut. Prinsip-prinsip ini memastikan bahwa perangkat lunak tidak dibatasi oleh bidang penggunaan, wilayah, atau pasar, sehingga mendorong kolaborasi dan transparansi dalam pengembangan [5].

On-Premises Software

On-Premises Software mengacu pada model penerapan perangkat lunak di mana perangkat lunak diinstal dan dijalankan pada komputer atau server yang berada di lokasi fisik [3].

Software Development Kit (SDK)

Software Development Kit (SDK) adalah kumpulan alat pengembangan perangkat lunak yang disediakan dalam satu paket untuk membantu pembuatan aplikasi pada platform tertentu [17]. SDK mempermudah proses pengembangan dengan menyediakan komponen yang sudah jadi dan panduan menggunakannya.

Unity Plugin

Unity Plugin adalah sekumpulan kode yang dibuat di luar Unity dan digunakan untuk menambahkan fungsionalitas pada proyek Unity. Terdapat dua jenis plugin di Unity: Managed Plugins dan Native Plugins. Managed Plugins adalah assembly .NET yang dikelola dan dibuat dengan alat seperti Visual Studio, sedangkan Native Plugins adalah perpustakaan kode asli yang spesifik untuk platform tertentu. Plugin memungkinkan akses ke fitur sistem operasi dan perpustakaan pihak ketiga yang biasanya tidak dapat diakses langsung oleh Unity [18].
